\documentclass[a4paper, 12pt]{article}
\usepackage[utf8]{inputenc}
\usepackage{amsmath}

% Encabezado
\title{Tarea 1: Álgebra Lineal}
\author{Facultad de Ingeniería Ambiental}
\date{Fecha de entrega: 14 de agosto de 2024}

\begin{document}

\maketitle

\section*{Instrucciones}
Resuelve los siguientes problemas de manera clara y completa. Asegúrate de incluir todos los pasos necesarios para llegar a la solución. Aplica los conceptos aprendidos en clase y justifica tus respuestas cuando sea necesario.

%\newpage

\section*{Problemas}

\begin{enumerate}
    \item Dada la matriz
    \[
    A = \begin{pmatrix}
    2 & -1 & 0 \\
    3 & 1 & 4 \\
    1 & 2 & 5
    \end{pmatrix},
    \]
    encuentra su adjunta y, si es posible, su inversa.
    \newline
    \textbf{Solución:} \\
    Primero, calculamos el determinante de \(A\). Si el determinante es diferente de cero, entonces la matriz es invertible. Luego, calculamos la adjunta y finalmente la inversa usando la fórmula \(\text{A}^{-1} = \frac{1}{\det(A)} \text{adj}(A)\).

    %\newpage

    \item Calcula la adjunta e inversa de la siguiente matriz:
    \[
    B = \begin{pmatrix}
    0 & 2 & -1 \\
    1 & 1 & 1 \\
    4 & -3 & 2
    \end{pmatrix}.
    \]
    
    \textbf{Solución:} \\
    Similar al problema anterior, comenzamos calculando el determinante de \(B\). Una vez que confirmamos que es diferente de cero, procedemos a encontrar la matriz adjunta de \(B\) y utilizamos la fórmula para calcular la inversa de \(B\).

    %\newpage
    
    \item Para la matriz
    \[
    C = \begin{pmatrix}
    1 & 2 & 3 \\
    0 & 1 & 4 \\
    5 & 6 & 0
    \end{pmatrix},
    \]
    encuentra la adjunta y determina si la matriz tiene inversa. En caso afirmativo, calcula la inversa.
    
    \textbf{Solución:} \\
    Primero, determinamos el determinante de la matriz \(C\). Si es diferente de cero, calculamos la matriz adjunta de \(C\) y finalmente la inversa usando \(\text{C}^{-1} = \frac{1}{\det(C)} \text{adj}(C)\).
    
    %\newpage
    
    \item Dada la matriz
    \[
    D = \begin{pmatrix}
    3 & 0 & 2 \\
    2 & 0 & -2 \\
    0 & 1 & 1
    \end{pmatrix},
    \]
    encuentra su adjunta y, si es posible, su inversa.
    
    \textbf{Solución:} \\
    Calculamos el determinante de \(D\) para verificar si es invertible. Luego, hallamos la adjunta y usamos la fórmula para calcular la inversa.

    % Puedes continuar agregando más problemas y soluciones en el mismo formato.

\end{enumerate}

\end{document}
