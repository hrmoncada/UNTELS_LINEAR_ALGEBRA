\documentclass[10pt,a4paper]{article}
\usepackage[utf8]{inputenc}
\usepackage{amsmath}
\usepackage{amsfonts}
\usepackage{amssymb}
\usepackage{fancyhdr}
\usepackage{multicol}
\usepackage{graphicx}
\usepackage{cancel}
\usepackage{pstricks}
\usepackage{enumitem}
% \topmargin = -0.5in      %
% \evensidemargin = -.1in     %
% \oddsidemargin = -.6in      %
% \textwidth = 7.6in        %
% \textheight = 9.5in       %
% \headsep = 0.25in
\linespread{1}
\usepackage
[
        a4paper,% other options: a3paper, a5paper, etc
        left=1cm,
        right=2cm,
        top=2cm,
        bottom=2cm,
        % use vmargin=2cm to make vertical margins equal to 2cm.
        % us  hmargin=3cm to make horizontal margins equal to 3cm.
        % use margin=3cm to make all margins  equal to 3cm.
]
{geometry}
% Encabezado
\title{Tarea 1: Álgebra Lineal}
\author{Facultad de Ingeniería Ambiental}
\date{Fecha de entrega: 14 de Setiembre de 2024}

\begin{document}
%--------------------------------------------------------------------------------------------------
% \centerline{\underline{\hspace{6in}}}
\maketitle
\centerline{\underline{\hspace{7in}}}
%--------------------------------------------------------------------------------------------------
\section*{Instrucciones}
Resuelve los siguientes problemas de manera clara y completa. Asegúrate de incluir todos los pasos necesarios para llegar a la solución. Aplica los conceptos aprendidos en clase y justifica tus respuestas cuando sea necesario.


\section*{Problemas :}
\begin{enumerate}
\item \textbf{Operaciones con Matrices}
Dadas las matrices $A = \begin{pmatrix} 1 & 2 \\ 3 & 4 \end{pmatrix}$ y $B = \begin{pmatrix} 2 & 0 \\ 1 & 3 \end{pmatrix}$, calcula $A + B$, $A - B$, y $AB$.

\item \textbf{Matriz Transpuesta}
Encuentra la matriz transpuesta de $C = \begin{pmatrix} 5 & -1 & 3 \\ 2 & 0 & 4 \end{pmatrix}$. ¿Es la matriz $C$ simétrica?

\item \textbf{Matriz Simétrica}
Demuestra si la matriz $D = \begin{pmatrix} 2 & 3 \\ 3 & 2 \end{pmatrix}$ es simétrica.

\item \textbf{Matriz Antisimétrica}
Determina si la matriz $E = \begin{pmatrix} 0 & 2 \\ -2 & 0 \end{pmatrix}$ es antisimétrica.

\item \textbf{Matriz Involutiva}
Verifica si la matriz $F = \begin{pmatrix} 0 & 1 \\ 1 & 0 \end{pmatrix}$ es involutiva, es decir, si cumple $F^2 = I$ donde $I$ es la matriz identidad.

\item \textbf{Matriz Idempotente}
Comprueba si la matriz $G = \begin{pmatrix} 2 & -2 \\ 1 & -1 \end{pmatrix}$ es idempotente, es decir, si cumple $G^2 = G$.

\item \textbf{Matriz Ortogonal}
Sea la matriz $H = \begin{pmatrix} \frac{1}{\sqrt{2}} & \frac{1}{\sqrt{2}} \\ -\frac{1}{\sqrt{2}} & \frac{1}{\sqrt{2}} \end{pmatrix}$. Verifica si $H$ es una matriz ortogonal.

\item \textbf{Producto Escalar}
Calcula el producto escalar de los vectores $\mathbf{u} = (1, 2, 3)$ y $\mathbf{v} = (4, 5, 6)$.

\item \textbf{Producto Vectorial}
Encuentra el producto vectorial de los vectores $\mathbf{a} = (1, 0, 0)$ y $\mathbf{b} = (0, 1, 0)$.

\item \textbf{Aplicación en Física}
Dada una matriz de rotación en 2D $R = \begin{pmatrix} \cos \theta & -\sin \theta \\ \sin \theta & \cos \theta \end{pmatrix}$, muestra que esta matriz es ortogonal y encuentra su transpuesta.
\end{enumerate}
% Continúa añadiendo problemas según sea necesario...
\section*{Instrucciones}
Resuelve los siguientes problemas sobre matrices de $3 \times 3$. Asegúrate de mostrar todos los pasos necesarios para llegar a la solución.

\begin{enumerate}[resume]
    %  1
    \item Sea la matriz $A = \begin{pmatrix} 1 & 2 & 3 \\ 4 & 5 & 6 \\ 7 & 8 & 9 \end{pmatrix}$. Encuentra la matriz transpuesta $A^T$.
%     \newline
%     \textbf{Solución:} \\
    %  2
    \item Determina si la matriz $B = \begin{pmatrix} 1 & 2 & 3 \\ 2 & 4 & 5 \\ 3 & 5 & 6 \end{pmatrix}$ es simétrica.
%     \newline
%     \textbf{Solución:} \\
    %  3
    \item Verifica si la matriz $C = \begin{pmatrix} 0 & 2 & -2 \\ -2 & 0 & 2 \\ 2 & -2 & 0 \end{pmatrix}$ es antisimétrica.
%     \newline
%     \textbf{Solución:} \\
    %  4
    \item Sea la matriz $D = \begin{pmatrix} 0 & 1 & 0 \\ -1 & 0 & 0 \\ 0 & 0 & -1 \end{pmatrix}$. Demuestra que es una matriz involutiva.

    %  5
    \item Encuentra si la matriz $E = \begin{pmatrix} 2 & -2 & 1 \\ 1 & 3 & -1 \\ -2 & 1 & 2 \end{pmatrix}$ es idempotente.

    %  6
    \item Verifica si la matriz $F = \begin{pmatrix} 1/\sqrt{2} & 0 & 1/\sqrt{2} \\ 0 & 1 & 0 \\ -1/\sqrt{2} & 0 & 1/\sqrt{2} \end{pmatrix}$ es ortogonal.

    %  7
    \item Calcula el producto escalar de los vectores fila de la matriz $G = \begin{pmatrix} 1 & 2 & 3 \\ 4 & 5 & 6 \\ 7 & 8 & 9 \end{pmatrix}$.

    %  8
    \item Realiza el producto vectorial de los vectores columna de la matriz $H = \begin{pmatrix} 1 & 0 & 0 \\ 0 & 1 & 0 \\ 0 & 0 & 1 \end{pmatrix}$.

    %  9
    \item Dada la matriz $I = \begin{pmatrix} 3 & -1 & 4 \\ 1 & 0 & 2 \\ -2 & 1 & 3 \end{pmatrix}$, encuentra el determinante de $I$.

    %  10
    \item Calcula la inversa de la matriz $J = \begin{pmatrix} 2 & 1 & 1 \\ 1 & 2 & 1 \\ 1 & 1 & 2 \end{pmatrix}$ si existe.

    %  11
    \item Sea la matriz $K = \begin{pmatrix} 5 & 2 & 3 \\ 2 & 5 & 4 \\ 3 & 4 & 5 \end{pmatrix}$. Determina si $K$ es positiva definida.

    %  12
    \item Verifica si la matriz $L = \begin{pmatrix} 1 & 0 & 0 \\ 0 & -1 & 0 \\ 0 & 0 & 1 \end{pmatrix}$ es ortogonal.

    %  13
    \item Encuentra el rango de la matriz $M = \begin{pmatrix} 1 & 2 & 3 \\ 4 & 5 & 6 \\ 7 & 8 & 9 \end{pmatrix}$.

    %  14
    \item Demuestra que la matriz $N = \begin{pmatrix} 0 & -1 & 0 \\ 1 & 0 & 0 \\ 0 & 0 & 1 \end{pmatrix}$ es una matriz de rotación.

    %  15
    \item Si $O = \begin{pmatrix} 2 & 0 & 0 \\ 0 & 2 & 0 \\ 0 & 0 & 2 \end{pmatrix}$, ¿qué tipo de matriz es $O$? Justifica tu respuesta.

    %  16
    \item Verifica si la matriz $P = \begin{pmatrix} 1 & 1 & 1 \\ 0 & 1 & 0 \\ 0 & 0 & 1 \end{pmatrix}$ es triangular superior.

    %  17
    \item Encuentra los autovalores de la matriz $Q = \begin{pmatrix} 2 & 0 & 0 \\ 0 & 3 & 0 \\ 0 & 0 & 4 \end{pmatrix}$.

    %  18
    \item Determina si la matriz $R = \begin{pmatrix} 0 & 1 & 2 \\ -1 & 0 & 3 \\ -2 & -3 & 0 \end{pmatrix}$ es una matriz antisimétrica.

    %  19
    \item Sea la matriz $S = \begin{pmatrix} 1 & 2 & 3 \\ 4 & 5 & 6 \\ 7 & 8 & 9 \end{pmatrix}$. Calcula $S^2$.

    %  20
    \item Encuentra el determinante de la matriz $T = \begin{pmatrix} 1 & 2 & 3 \\ 0 & 1 & 4 \\ 5 & 6 & 0 \end{pmatrix}$.

\end{enumerate}


\section*{Instrucciones}
Resuelva los siguientes problemas de productos escalar y vectorial utilizando matrices de 3x3.

\section*{Problemas}

\begin{enumerate}[resume]
    % Problema 1
    \item Dados los vectores $\mathbf{a} = \begin{bmatrix} 1 \\ 2 \\ 3 \end{bmatrix}$ y $\mathbf{b} = \begin{bmatrix} 4 \\ 5 \\ 6 \end{bmatrix}$, calcule el producto escalar $\mathbf{a} \cdot \mathbf{b}$.
%     \newline
%     \textbf{Solución:} \\
    % Problema 2
    \item Dados los vectores $\mathbf{c} = \begin{bmatrix} 2 \\ -1 \\ 3 \end{bmatrix}$ y $\mathbf{d} = \begin{bmatrix} 0 \\ 5 \\ -2 \end{bmatrix}$, encuentre el producto vectorial $\mathbf{c} \times \mathbf{d}$.
%     \newline
%     \textbf{Solución:} \\
    % Problema 3
    \item Encuentre el producto escalar de los vectores $\mathbf{e} = \begin{bmatrix} -2 \\ 4 \\ 1 \end{bmatrix}$ y $\mathbf{f} = \begin{bmatrix} 3 \\ -6 \\ 2 \end{bmatrix}$.
%     \newline
%     \textbf{Solución:} \\
    % Problema 4
    \item Si $\mathbf{g} = \begin{bmatrix} 1 \\ 2 \\ 3 \end{bmatrix}$ y $\mathbf{h} = \begin{bmatrix} -1 \\ 0 \\ 4 \end{bmatrix}$, calcule $\mathbf{g} \times \mathbf{h}$.

    % Problema 5
    \item Dado $\mathbf{i} = \begin{bmatrix} 5 \\ -3 \\ 2 \end{bmatrix}$ y $\mathbf{j} = \begin{bmatrix} 2 \\ 4 \\ -1 \end{bmatrix}$, encuentre el producto escalar $\mathbf{i} \cdot \mathbf{j}$ y el producto vectorial $\mathbf{i} \times \mathbf{j}$.

    % Problema 6
    \item Calcule el producto escalar de $\mathbf{k} = \begin{bmatrix} 6 \\ 7 \\ -2 \end{bmatrix}$ y $\mathbf{l} = \begin{bmatrix} -3 \\ 2 \\ 5 \end{bmatrix}$.

    % Problema 7
    \item Encuentre el producto vectorial de $\mathbf{m} = \begin{bmatrix} 1 \\ -4 \\ 2 \end{bmatrix}$ y $\mathbf{n} = \begin{bmatrix} 3 \\ 0 \\ 1 \end{bmatrix}$.

    % Problema 8
    \item Dados los vectores $\mathbf{o} = \begin{bmatrix} 7 \\ 1 \\ -5 \end{bmatrix}$ y $\mathbf{p} = \begin{bmatrix} -2 \\ 3 \\ 4 \end{bmatrix}$, calcule $\mathbf{o} \cdot \mathbf{p}$.

    % Problema 9
    \item Si $\mathbf{q} = \begin{bmatrix} 0 \\ -2 \\ 5 \end{bmatrix}$ y $\mathbf{r} = \begin{bmatrix} 2 \\ 1 \\ -3 \end{bmatrix}$, encuentre el producto vectorial $\mathbf{q} \times \mathbf{r}$.

    % Problema 10
    \item Encuentre el producto escalar y vectorial de los vectores $\mathbf{s} = \begin{bmatrix} 2 \\ 3 \\ -1 \end{bmatrix}$ y $\mathbf{t} = \begin{bmatrix} 4 \\ -2 \\ 0 \end{bmatrix}$.
\end{enumerate}

% \section*{Soluciones}
% 
% \begin{enumerate}[resume]
%     % Solución 1
%     \item $\mathbf{a} \cdot \mathbf{b} = 1\cdot4 + 2\cdot5 + 3\cdot6 = 4 + 10 + 18 = 32$
% 
%     % Solución 2
%     \item $\mathbf{c} \times \mathbf{d} = \begin{vmatrix}
%     \mathbf{i} & \mathbf{j} & \mathbf{k} \\
%     2 & -1 & 3 \\
%     0 & 5 & -2 
%     \end{vmatrix} = \mathbf{i}((-1)(-2) - (3)(5)) - \mathbf{j}((2)(-2) - (3)(0)) + \mathbf{k}((2)(5) - (-1)(0))$
% 
%     $\mathbf{c} \times \mathbf{d} = \mathbf{i}(2 - 15) - \mathbf{j}(-4) + \mathbf{k}(10)$
% 
%     $\mathbf{c} \times \mathbf{d} = \begin{bmatrix} -13 \\ 4 \\ 10 \end{bmatrix}$
% 
%     % Añade soluciones para los demás problemas aquí.
% \end{enumerate}
\section*{Instrucciones}
A continuación, encontrarás 10 problemas que te ayudarán a practicar el cálculo de la adjunta e inversa de matrices de 3x3. Para cada problema, se proporcionará una matriz específica. Debes calcular su adjunta y, si es invertible, su inversa. 

\section*{Problemas}

\begin{enumerate}[resume]
    \item Dada la matriz
    \[
    A = \begin{pmatrix}
    2 & -1 & 0 \\
    3 & 1 & 4 \\
    1 & 2 & 5
    \end{pmatrix},
    \]
    encuentra su adjunta y, si es posible, su inversa.
%     \newline
%     \textbf{Solución:} \\    
    \item Calcula la adjunta e inversa de la siguiente matriz:
    \[
    B = \begin{pmatrix}
    0 & 2 & -1 \\
    1 & 1 & 1 \\
    4 & -3 & 2
    \end{pmatrix}.
    \]
%     \newline
%     \textbf{Solución:} \\    
    \item Para la matriz
    \[
    C = \begin{pmatrix}
    1 & 2 & 3 \\
    0 & 1 & 4 \\
    5 & 6 & 0
    \end{pmatrix},
    \]
    encuentra la adjunta y determina si la matriz tiene inversa. En caso afirmativo, calcula la inversa.
%     \newline
%     \textbf{Solución:} \\    
    \item Dada la matriz
    \[
    D = \begin{pmatrix}
    3 & 0 & 2 \\
    2 & 0 & -2 \\
    0 & 1 & 1
    \end{pmatrix},
    \]
    encuentra su adjunta y, si es posible, su inversa.
    
    \item Calcula la adjunta e inversa de la siguiente matriz:
    \[
    E = \begin{pmatrix}
    1 & 0 & 1 \\
    2 & 1 & 0 \\
    3 & 2 & 1
    \end{pmatrix}.
    \]
    
    \item Para la matriz
    \[
    F = \begin{pmatrix}
    2 & 5 & 7 \\
    6 & 3 & 4 \\
    5 & -2 & -3
    \end{pmatrix},
    \]
    encuentra la adjunta y determina si la matriz tiene inversa. En caso afirmativo, calcula la inversa.
    
    \item Dada la matriz
    \[
    G = \begin{pmatrix}
    0 & -2 & 1 \\
    1 & 0 & 2 \\
    4 & 1 & 0
    \end{pmatrix},
    \]
    encuentra su adjunta y, si es posible, su inversa.
    
    \item Calcula la adjunta e inversa de la siguiente matriz:
    \[
    H = \begin{pmatrix}
    4 & 1 & 0 \\
    -1 & 2 & 3 \\
    2 & -2 & 1
    \end{pmatrix}.
    \]
    
    \item Para la matriz
    \[
    I = \begin{pmatrix}
    3 & 1 & -1 \\
    1 & 3 & -1 \\
    0 & 0 & 1
    \end{pmatrix},
    \]
    encuentra la adjunta y determina si la matriz tiene inversa. En caso afirmativo, calcula la inversa.
    
    \item Dada la matriz
    \[
    J = \begin{pmatrix}
    1 & 2 & 1 \\
    0 & 1 & 2 \\
    1 & 0 & 1
    \end{pmatrix},
    \]
    encuentra su adjunta y, si es posible, su inversa.
\end{enumerate}
\end{document}
